\linespread{1.3}
\usepackage[margin=1in]{geometry}
\geometry{a4paper,top=3cm,bottom=3cm,left=3.5cm,right=3.5cm,heightrounded,bindingoffset=5mm}
%%%% PACCHETTI
\usepackage{
	amsmath,
	amsfonts,
	amssymb,
	amsthm,
	epstopdf,
	epigraph,
	yfonts,
	xcolor,
	graphicx,
	dsfont,
	newlfont,
	emptypage,
	listings,
	color,
	booktabs,	
	subfiles,
	lmodern,
	fancyhdr,
	titlesec,
	tikz,
	thmtools,
	stmaryrd,
	}

%% the palette is https://www.colourlovers.com/palette/4648310/Mild_Mannered
\definecolor{circusroyale4}{HTML}{803C37}
\definecolor{everglow}{HTML}{3F5E58}
	
\usepackage[T1]{fontenc}
\usepackage[utf8]{inputenc}
\usepackage{eulervm, palatino}
\usepackage[colorlinks=true,allcolors=circusroyale4]{hyperref}
\usepackage[tocgraduated]{tocstyle}
\usetocstyle{allwithdot}

%%%% THEOREMS
\newtheoremstyle{custom-up}% name of the style to be used
{6pt}% measure of space to leave above the theorem. E.g.: 3pt
{3pt}% measure of space to leave below the theorem. E.g.: 3pt
{}% name of font to use in the body of the theorem
{}% measure of space to indent
{\scshape}% name of head font
{}% punctuation between head and body
{2pt}% space after theorem head; " " = normal interword space
{\indent{\thmname{#1} \hskip0.8mm\textbf{\textsf{\footnotesize{\textcolor{circusroyale4}{\thmnumber{#2}}}}}}\hskip1.5mm{\thmnote{(#3)\hskip0.5mm}}}%

\newtheoremstyle{custom-wt}% name of the style to be used
{6pt}% measure of space to leave above the theorem. E.g.: 3pt
{3pt}% measure of space to leave below the theorem. E.g.: 3pt
{}% name of font to use in the body of the theorem
{}% measure of space to indent
{\scshape}% name of head font
{}% punctuation between head and body
{3pt}% space after theorem head; " " = normal interword space
{{\thmname{#1}}\hskip1mm\thmnote{(#3)}}%


\newtheoremstyle{custom-it}% name of the style to be used
{6pt}% measure of space to leave above the theorem. E.g.: 3pt
{}% measure of space to leave below the theorem. E.g.: 3pt
{\itshape}% name of font to use in the body of the theorem
{}% measure of space to indent
{\scshape}% name of head font
{}% punctuation between head and body
{2pt}% space after theorem head; " " = normal interword space
{\indent{\thmname{#1} \hskip0.8mm\textbf{\textsf{\footnotesize{\textcolor{circusroyale4}{\thmnumber{#2}}}}}}\hskip1.5mm{\thmnote{(#3)\hskip0.5mm}}}%

\theoremstyle{custom-up}
\newtheorem*{axiom}{axiom}
\newtheorem*{question}{question}
\newtheorem{definition}{definition}[section]
\newtheorem*{cpp}{CPP}
\newtheorem*{alp}{ALP}


\theoremstyle{custom-wt}
\newtheorem*{example}{example}
\newtheorem*{remark}{remark}
\newtheorem*{claim}{claim}


\theoremstyle{custom-it}
\newtheorem{theorem}[definition]{theorem}
\newtheorem{corollary}[definition]{corollary} 
\newtheorem{lemma}[definition]{lemma} 
\newtheorem{proposition}[definition]{proposition} 
\newtheorem*{theorem*}{theorem}

%%%% TITLESEC
\newcommand{\hsp}{\hspace{10pt}}
\titleformat{\chapter}[hang]{\Huge\bfseries}{\textcolor{circusroyale4}{\thechapter}\hsp}{0pt}{\Huge}
\titleformat{\section}[hang]{\Large\bfseries}{\textcolor{circusroyale4}{\thesection}\hsp{|}\hsp}{0pt}{\Large\bfseries}
\titleformat{\subsection}[hang]{\large\bfseries}{\textcolor{circusroyale4}{\thesubsection}\hsp{|}\hsp}{0pt}{\large\scshape\MakeLowercase}


%% quanta fatica per lo spazio sopra proof
\declaretheoremstyle[%
spaceabove=-4pt,%
spacebelow=6pt,%
headfont=\normalfont\scshape,%
postheadspace=3pt,%
qed=\qedsymbol%
]{prfstyle} 
\declaretheorem[name={proof},style=prfstyle,unnumbered,
]{prf}

\renewenvironment{proof}{\begin{prf}}{%
	\end{prf}\ignorespacesafterend
}

%%% FONTS
\newcommand{\bb}[1]{\mathbb{#1}}
\newcommand{\ds}[1]{\mathds{#1}}
\newcommand{\fk}[1]{\mathfrak{#1}}
\newcommand{\sff}[1]{\mathsf{#1}}
\newcommand{\cl}[1]{\mathcal{#1}}

%%%% SYMBOLS
\newcommand{\comp}[1]{#1^{\mathsf{c}}}
\newcommand{\powerset}[1]{\cl{P}(#1)}
\newcommand{\finset}[1]{\cl{P}_{\mathsf{fin}}(#1)}
\newcommand{\struct}[1]{\mathds{#1}} % use this for "famous" sets and the like
\newcommand{\sm}{\smallsetminus}
\newcommand{\quot}[2]{{#1}/{#2}}
\newcommand{\wt}[1]{\widetilde{#1}}
\newcommand{\into}{\hookrightarrow}
\newcommand{\then}{\rightarrow}
\newcommand{\meet}{\land}
\newcommand{\join}{\lor}
\newcommand{\onto}{\twoheadrightarrow}

%% STRUCTURES
\newcommand{\NN}{\struct{N}}
\newcommand{\QQ}{\struct{Q}}
\newcommand{\ZZ}{\struct{Z}}
\newcommand{\RR}{\struct{R}}
\newcommand{\CC}{\struct{C}}
\newcommand{\Q}[1]{\struct{Q}_{#1}}
\newcommand{\Z}[1]{\struct{Z}_{#1}}

%%% MISC
\renewcommand{\emph}[1]{\textbf{#1}}
\renewcommand{\geq}{\geqslant}
\renewcommand{\leq}{\leqslant}
\renewcommand{\phi}{\varphi}
\renewcommand{\epsilon}{\varepsilon}
\newcommand{\interior}[1]{{#1}^{\circ}}

%% TESI
\usepackage[backend=biber,style=alphabetic,sorting=ynt]{biblatex}
\usepackage{hyperref}
\addbibresource{biblio.bib}
\usepackage[tocgraduated]{tocstyle}
\usetocstyle{allwithdot}

\newcommand{\ns}[1]{{{}^{*}\!{#1}}}
\newcommand{\completion}{\mathbb{R}_{\Lambda}}
\newcommand{\lambdalim}[1]{\lim\limits_{\lambda \uparrow \Lambda}{#1}}
\newcommand{\tbd}{\colorbox{circusroyale4!80}{\textbf{\textsf{TBD}}}}
\newcommand{\prob}{\textbf{P}}

%%%% fancyhdr
\pagestyle{fancy}
\fancyhead[LE,RO]{\textit{Through the $\Lambda$ glass}}
\fancyhead[LO,RE]{\leftmark}
\fancyfoot[C]{\thepage}

